% Tikz-timing example of PCI interrupt acknowledge.
%
% Author: Nathan Typanski <http://nathantypanski.com>
% License: Creative Commons Attribution 4.0 International License
%          <http://creativecommons.org/licenses/by/4.0/>

\documentclass{standalone}

\usepackage{xparse} % NewDocumentCommand, IfValueTF, IFBooleanTF

\usepackage{tikz-timing}[2014/10/29]
\usetikztiminglibrary[rising arrows]{clockarrows}

% Reference a bus.
%
% Usage:
%
%     \busref[3::0]{C/BE}    ->   C/BE[3::0]
%     \busref*{AD}           ->   AD#
%     \busref*[3::0]{C/BE}   ->   C/BE[3::0]#
%
\NewDocumentCommand{\busref}{som}{\texttt{%
#3%
\IfValueTF{#2}{[#2]}{}%
\IfBooleanTF{#1}{\#}{}%
}}

\begin{document}

\begin{tikztimingtable}[%
    timing/dslope=0.1,
    timing/.style={x=5ex,y=2ex},
    x=5ex,
    timing/rowdist=3ex,
    timing/name/.style={font=\sffamily\scriptsize}
]
\busref{CLK}         & 10{C} \\
\busref*{FRAME}      & U h l L l h 4H 2U \\
\busref[31::0]{AD}   & U u 2X 2.5U 2D{$v_i$} 2U \\
\busref*[3::0]{C/BE} & U u 2D{0000} 4.5D{\busref*{BE}} 2U  \\
\busref*{IRDY}       & 3.5U 4.5L 2H \\
\busref*{TRDY}       & 3.5U 2.5H 2L 2H \\
\busref*{DEVSEL}     & 5U hl 2L h 1.5U \\
\extracode
\begin{pgfonlayer}{background}
\begin{scope}[semitransparent ,semithick]
\vertlines[darkgray,dotted]{1.0,3.0,...,9.0}
\vertlines[gray,dotted]{2.0,4.0,...,8.0}
\end{scope}
\end{pgfonlayer}
\end{tikztimingtable}

\end{document}
